% !TEX program = xelatex

\documentclass{resume}
%\usepackage{zh_CN-Adobefonts_external} % Simplified Chinese Support using external fonts (./fonts/zh_CN-Adobe/)
%\usepackage{zh_CN-Adobefonts_internal} % Simplified Chinese Support using system fonts

\begin{document}
\pagenumbering{gobble} % suppress displaying page number

\name{Xingguo Jia}

\basicInfo{
  \email{jiaxg1998@sjtu.edu.com} }
  % \phone{(+86) 131-221-87xxx} \textperiodcentered\ 
  % \linkedin[billryan8]{https://www.linkedin.com/in/billryan8}}

\section{\faGraduationCap\ Education}
\datedsubsection{\textbf{Shanghai Jiao Tong University (SJTU)}, Shanghai, China}{2020 -- Present}
\textit{Master student} in Software Engineering (SE), expected March 2023
\begin{itemize}
  \item Advisor: Prof. Zhengwei Qi
\end{itemize}
\datedsubsection{\textbf{Shanghai Jiao Tong University (SJTU)}, Shanghai, China}{2016 -- 2020}
\textit{B.E.} in Software Engineering (SE)

\section{\faUsers\ Research Experiences}
\datedsubsection{\textbf{SJTU}, Lab of System Control and Information Processing}{Sep. 2017 -- Jun. 2019}
\role{Differential Privacy}{}
Brief introduction:
\begin{itemize}
  \item Designing an algorithm for nodes in a cluster to reach a consensus on some status, such as clock.
\end{itemize}

\datedsubsection{\textbf{SJTU}, Trusted Cloud Group (TCloud)}{Jun. 2019 -- Present}
\role{GiantVM}{}
Brief introduction:
\begin{itemize}
  \item Optimizing GiantVM performance when running memory-intensive applications.
\end{itemize}

% Reference Test
%\datedsubsection{\textbf{Paper Title\cite{zaharia2012resilient}}}{May. 2015}
%An xxx optimized for xxx\cite{verma2015large}
%\begin{itemize}
%  \item main contribution
%\end{itemize}

\section{\faInfo\ Projects}
\datedsubsection{\textbf{GiantVM: DSM-aware Scheduler}}{Aug. 2020 -- Oct. 2021}
\textbf{Link:} https://github.com/GiantVM/Linux-DSM/tree/taco-eval

On GiantVM, memory-intensive applications could suffer from cross-node page sharing, which requires frequent DSM involvement and leads to performance collapse. We design the guest-level thread scheduler, DaS (DSM-aware Scheduler), to overcome the bottleneck. When benchmarking with NPB, the DaS could achieve a performance boost of up to 3.5$\times$, compared to the default Linux kernel scheduler. 

\datedsubsection{\textbf{YFS: Yet another File System}}{Sep. 2018 -- Jan. 2019}
\textbf{Link:} https://github.com/snake0/lab-cse

Implement a distributed file system YFS based on \texttt{inode} and \texttt{RPC} in C++. We first implement a single-node version of YFS that supports sequential read/write operations, then extend it to be distributed using RPC. Finally we implement a lock server that could batch locking requests for concurrent file system accesses.

\section{\faGraduationCap\ Publications}
\begin{itemize}
\item Jin Zhang, Zhuocheng Ding, Yubin Chen, \textbf{Xingguo Jia}, Boshi Yu, Zhengwei Qi, and Haibing Guan. 2020. GiantVM: a Type-II Hypervisor Implementing Many-to-one Virtualization. In \textit{VEE'20}
\item \textbf{Xingguo Jia}, Jin Zhang, Boshi Yu, Xingyue Qian, Zhengwei Qi, and Haibing Guan. 2021. GiantVM: A Novel Distributed Hypervisor for Resource Aggregation with DSM-aware Optimizations. In \textit{TACO'22} (To appear)
\end{itemize}

\section{\faCogs\ Skills}
\begin{itemize}[parsep=0.5ex]
  \item Programming Languages: C == Python > C++ > Shell
  \item Tools: \LaTeX, Git, GDB, HTML, CSS
  \item Systems: Linux, QEMU
\end{itemize}

\section{\faHeartO\ Honors and Awards}
\datedline{\textit{First-class} academic scholarship, SJTU }{Dec. 2021}
\datedline{Successful Participant, MCM }{Jan. 2019}

\section{\faUsers\ Teaching Experiences}
\datedline{Advanced Cloud Operating Systems, SJTU }{Dec. 2021}
\datedline{Advanced Data Structure, SJTU }{Jun. 2021}

\section{\faInfo\ Miscellaneous}
\begin{itemize}[parsep=0.5ex]
  \item GiantVM: https://giantvm.github.io/
  \item GitHub: https://github.com/snake0
  \item Languages: English - Fluent, Mandarin - Native speaker
\end{itemize}

%% Reference
%\newpage
%\bibliographystyle{IEEETran}
%\bibliography{mycite}
\end{document}

