% !TEX TS-program = xelatex
% !TEX encoding = UTF-8 Unicode
% !Mode:: "TeX:UTF-8"

\documentclass{resume}
\usepackage{zh_CN-Adobefonts_external} % Simplified Chinese Support using external fonts (./fonts/zh_CN-Adobe/)
% \usepackage{NotoSansSC_external}
% \usepackage{NotoSerifCJKsc_external}
% \usepackage{zh_CN-Adobefonts_internal} % Simplified Chinese Support using system fonts
\usepackage{linespacing_fix} % disable extra space before next section
\usepackage{cite}

\begin{document}
\pagenumbering{gobble} % suppress displaying page number

\name{贾\ 兴\ 国}

\basicInfo{
  \email{jiaxg1998@sjtu.edu.cn} \textperiodcentered \
  \phone{(+86) 132-6262-6313} \textperiodcentered\ 
  \faGithub\ {https://github.com/snake0}}

  % \linkedin[billryan8]{https://www.linkedin.com/in/billryan8}}
 
\section{\faGraduationCap\  教育背景}
\datedsubsection{\textbf{上海交通大学(SJTU)}, 上海}{2020 -- 至今}
\textit{在读硕士研究生}\ 软件工程, 预计 2023 年 3 月毕业,\textbf{GPA:} 3.83/4
\begin{itemize}
  \item 上海市可扩展计算与系统重点实验室 (TCloud)
  \item \textbf{研究方向:} 系统虚拟化、操作系统、分布式系统  \ \  \textbf{导师:} 戚正伟教授
\end{itemize}
\datedsubsection{\textbf{上海交通大学(SJTU)}, 上海}{2016 -- 2020}
\textit{工学学士}\ 软件工程, \textbf{GPA:} 3.50/4.30
\begin{itemize}
  \item \textbf{学习方向:} 系统软件
\end{itemize}

\section{\faUsers\ 研究经历}
\datedsubsection{\textbf{TCloud} SJTU, 上海}{2019年1月 -- 至今}
{\textit{GiantVM:}} 自底向上性能测试与分析,基于线程调度的访存性能优化

\datedsubsection{\textbf{华为杭州研究所}, 操作系统开发实习生, 杭州}{2022年2月 --至今}
{\textit{Memory Pool:} 基于RDMA的集群内存池化方案调研与设计

\section{\faInfo\ 项目经历}
\datedsubsection{{\textbf{DaS:} 基于GiantVM 的 DSM-aware 调度器}\ \ \href{https://github.com/GiantVM/Linux-DSM/tree/taco-eval}{\faGithubSquare}}{2020年8月 -- 2021年10月}

在 GiantVM 上,内存密集型应用会造成节点之间的内存页共享,导致应用性能严重下降。于是,我设计了DSM-aware 调度器(DaS),通过调度算法降低线程间的跨节点内存页共享。在 GiantVM 客户机中运行 NPB 内存密集型应用时,DaS相比Linux内核默认调度器提升应用性能 3.5 倍。

\datedsubsection{{\textbf{LaS:} 基于 GiantVM 的集群任务迁移机制}\ \ \href{https://github.com/snake0/bachelor-thesis}{\faGithubSquare}}{2020年4月 -- 2020年8月}

数据中心一般通过虚拟机热迁移提升 CPU 使用率、保证服务QoS,但虚拟机热迁移存在网络带宽占用大、服务下线时间较长的问题。而GiantVM的分布式共享内存协议允许服务在目的节点缓存内存页。我利用这一优势设计了基于GiantVM的集群Load-aware调度机制(LaS),优化了高负载下集群服务的QoS,相比虚拟机热迁移降低了网络开销68\%。

\datedsubsection{{\textbf{kRDMA:} 基于 IB verbs 的内核RDMA通信接口}\ \ \href{https://github.com/snake0/krdma}{\faGithubSquare}}{2022年4月 -- 2022年6月}

IB verbs接口与传统的socket网络通信接口差别较大,编程难度较高。我对RDMA的四种原语进行封装,形成与socket相同的接口,并通过注册物理地址的方式降低RDMA网卡的缓存压力,优化性能。

\section{\faGraduationCap\ 发表论文}
\begin{itemize}
\item Jin Zhang, Zhuocheng Ding, Yubin Chen, \textbf{Xingguo Jia}, Boshi Yu, Zhengwei Qi, and Haibing Guan. 2020. GiantVM: a Type-II Hypervisor Implementing Many-to-one Virtualization. In \textbf{VEE'20}
\item \textbf{Xingguo Jia$^*$}, Jin Zhang$^*$, Boshi Yu, Xingyue Qian, Zhengwei Qi, and Haibing Guan. 2021. GiantVM: A Novel Distributed Hypervisor for Resource Aggregation with DSM-aware Optimizations. In \textbf{TACO'22}
\end{itemize}

% Reference Test
%\datedsubsection{\textbf{Paper Title\cite{zaharia2012resilient}}}{May. 2015}
%An xxx optimized for xxx\cite{verma2015large}
%\begin{itemize}
%  \item main contribution
%\end{itemize}

\section{\faCogs\ 技能}
% increase linespacing [parsep=0.5ex]
\begin{itemize}[parsep=0.5ex]
  \item \textbf{编程语言:} C > C++ > Python > Shell \ \ \textbf{工具:} \LaTeX, Git, GDB \ \ \textbf{系统:} Linux,QEMU
\end{itemize}

\section{\faHeartO\ 获奖情况}
%\datedline{成功参与奖,美国大学生数学建模竞赛(MCM)}{2019 年1 月}
\datedline{一等学业奖学金,SJTU}{2021 年12 月}
\datedline{《深入浅出系统虚拟化:原理与实践》内存虚拟化,教育部智能基座“突出贡献奖”}{2022 年1 月}

%% Reference
%\newpage
%\bibliographystyle{IEEETran}
%\bibliography{mycite}
\end{document}
