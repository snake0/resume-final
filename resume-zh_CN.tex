% !TEX TS-program = xelatex
% !TEX encoding = UTF-8 Unicode
% !Mode:: "TeX:UTF-8"

\documentclass{resume}
\usepackage{zh_CN-Adobefonts_external} % Simplified Chinese Support using external fonts (./fonts/zh_CN-Adobe/)
% \usepackage{NotoSansSC_external}
% \usepackage{NotoSerifCJKsc_external}
% \usepackage{zh_CN-Adobefonts_internal} % Simplified Chinese Support using system fonts
\usepackage{linespacing_fix} % disable extra space before next section
\usepackage{cite}

\begin{document}
\pagenumbering{gobble} % suppress displaying page number

\name{贾\ 兴\ 国}

\basicInfo{
  \email{jiaxg1998@sjtu.edu.cn} \textperiodcentered \
  \phone{(+86) 132-6262-6313} \textperiodcentered\ 
  \faGithub\ {https://github.com/snake0}}

  % \linkedin[billryan8]{https://www.linkedin.com/in/billryan8}}
 
\section{\faGraduationCap\  教育背景}
\datedsubsection{\textbf{上海交通大学(SJTU)}, 上海}{2020 -- 至今}
\textit{在读硕士研究生}\ 软件工程, 预计 2023 年 3 月毕业
\begin{itemize}
  \item 上海市可扩展计算与系统重点实验室
  \item \textbf{研究方向:} 系统虚拟化、操作系统  \ \  \textbf{导师:} 戚正伟教授
\end{itemize}
\datedsubsection{\textbf{上海交通大学(SJTU)}, 上海}{2016 -- 2020}
\textit{工学学士}\ 软件工程

\section{\faUsers\ 研究经历}
% \datedsubsection{\textbf{SJTU}, 系统控制与信息处理实验室}{2017年9月 --  2019年6月}
% \textbf{差分隐私:} 设计一个算法,使得集群中的节点快速达成对于某种状态的一致(如时钟)

\datedsubsection{\textbf{SJTU}, 上海市可扩展计算与系统重点实验室}{2019年6月 -- 至今}
\textit{GiantVM:} 完成GiantVM宏观、微观性能测试,并设计实现性能优化方案

\section{\faInfo\ 项目经历}
\datedsubsection{\textbf{GiantVM: DSM-aware 调度器}}{2020年8月 -- 2021年10月}
\textbf{链接}: https://github.com/GiantVM/Linux-DSM/tree/taco-eval

在 GiantVM 上,内存密集型应用会造成节点之间的内存页共享,从而需要频繁调用 DSM 模块,导致应用性能严重下降。于是,我设计了一个客户机操作系统中的 DSM-aware 调度器(简称 DaS),通过调整线程的调度决策,使得访问大量共享内存页的两个线程运行在 GiantVM 的同一个物理节点上。在 GiantVM 客户机中运行 NPB 内存密集型应用时,DaS可将应用性能相比Linux内核默认调度器提升 3.5 倍。我完成了所有代码的编写,并完成了性能测试。

\datedsubsection{\textbf{YFS: Yet another File System}}{2018年9月 -- 2019年1月}
\textbf{链接:} https://github.com/snake0/lab-cse

使用C++实现了分布式文件系统YFS。首先完成了一个基于\texttt{inode}的单节点文件系统,支持基础的操作,如创建、读写文件,又使用RPC将其扩展为分布式文件系统,最终实现锁服务器,减少了支持并行访问所需要的加锁、放锁操作。


\section{\faGraduationCap\ 发表论文}
\begin{itemize}
\item Jin Zhang, Zhuocheng Ding, Yubin Chen, \textbf{Xingguo Jia}, Boshi Yu, Zhengwei Qi, and Haibing Guan. 2020. GiantVM: a Type-II Hypervisor Implementing Many-to-one Virtualization. In \textbf{VEE'20}
\item \textbf{Xingguo Jia}, Jin Zhang, Boshi Yu, Xingyue Qian, Zhengwei Qi, and Haibing Guan. 2021. GiantVM: A Novel Distributed Hypervisor for Resource Aggregation with DSM-aware Optimizations. In \textbf{TACO'22} (To appear)
\end{itemize}

% Reference Test
%\datedsubsection{\textbf{Paper Title\cite{zaharia2012resilient}}}{May. 2015}
%An xxx optimized for xxx\cite{verma2015large}
%\begin{itemize}
%  \item main contribution
%\end{itemize}

\section{\faCogs\ 技能}
% increase linespacing [parsep=0.5ex]
\begin{itemize}[parsep=0.5ex]
  \item \textbf{编程语言:} C == Python > C++ > Shell \ \ \textbf{工具:} \LaTeX, Git, GDB \ \ \textbf{系统:} Linux 
\end{itemize}

\section{\faHeartO\ 获奖情况}
\datedline{一等学业奖学金,SJTU}{2021 年12 月}
\datedline{编写《深入浅出系统虚拟化:原理与实践》内存部分,获得教育部智能基座“突出贡献奖”}{2022 年1 月}

%% Reference
%\newpage
%\bibliographystyle{IEEETran}
%\bibliography{mycite}
\end{document}
